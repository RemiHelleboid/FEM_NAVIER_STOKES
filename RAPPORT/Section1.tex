\section{Équations de Stokes et Navier-Stokes}

\subsection{Équation de Navier-Stokes}
\subsubsection{Notations}
Soit $\Omega$ un ouvert de $\mathbf{R}^2$ ou $\mathbf{R}^3 $. \\
On note $\rho(t, \textbf{x})$ la densité du fluide au point $\textbf{x} \in \Omega$ et au temps $t$.\\
Et $\textbf{u}(t, \textbf{x})$ le vecteur vitesse du fluide au point $\textbf{x}$ et au temps $t$. \\
$\textbf{n}(\textbf{x}) $ est le vecteur normal sortant de l’interface considéré.\\
$\mathbf{F}$ désigne une force s'appliquant au fluide. \\
$\mathbf{a}(t, \mathbf{x})$ est l'accélération du fluide au point $\textbf{x}$ et au temps $t$. \\
\subsubsection{Équation de conservation}
On fait l'hypothèse que la masse du fluide est toujours conservée. La masse du fluide est égale à la densité volumique multipliée par le volume, autrement dit c'est l'intégrale sur un volume (ou une surface) $\mathcal{V}$ de la densité $\rho(\cdot, t)$.
Sa conservation signifie que la variation de la masse dans $\mathcal{V}$ est strictement égale à la somme de ce qui entre dans le volume $\mathcal{V}$ et de ce qui en sort.
Or à l'interface du $\mathcal{V}$, la masse de fluide sortant (ou entrant) est égale à la vitesse du fluide projeté sur le vecteur sortant de la surface, multiplié par sa densité, le tout multiplié par la source totale de l'interface de $\mathcal{V}$.
Si on fait un raisonnement par grandeur on obtient que la variation de masse est en :
$\left[ kg.m^{-3} \cdot m.s^{-1} \cdot m^2 \right] = \left[ kg.s^{-1} \right] $ ce qui correspond bien à une variation temporelle de masse.
On peut donc écrire en termes mathématiques : 

\begin{equation}\label{MassCons}
\frac{d}{d t}\left(\int_{\mathcal{V}} \rho(t, \mathbf{x}) \mathrm{d} x\right)=-\int_{\partial \mathcal{V}} \rho(t, \mathbf{x}) \mathbf{u}(t, \mathbf{x}) \cdot \mathbf{n}(\mathbf{x}) \mathrm{d}s
\end{equation}

On peut appliquer la formule de Stokes : 
\begin{equation}\label{StokesFormula}
\int_{\partial \mathcal{V}} \mathbf{v} \cdot \mathbf{n} \mathrm{d} s=\int_{\mathcal{V}} \operatorname{div} \mathbf{v} \mathrm{d} x
\end{equation}

au second membre de l'équation \ref{MassCons}, pour obtenir : 
\begin{equation}
\int_{\mathcal{V}} \frac{\partial\rho}{\partial t} + div(\rho \mathbf{u}) \mathrm{d} x = 0
\end{equation}

Considérant que les fonctions sont assez régulière on a également effectué une permutation d'intégrale et de dérivée dans le membre de gauche.

Comme $\mathcal{V}$ est quelconque on peut se ramener à une expression locale : 
\begin{equation}\label{MassConsLoc}
\frac{\partial\rho}{\partial t} + div(\rho \mathbf{u})  = 0 
\end{equation}
Pour tout $t \in [0, T] $ et $x \in \Omega$.

\subsubsection{Seconde loi de Newton}
La seconde loi de Newton nous dit que :
\[\sum \mathbf{F} = m \mathbf{a}\]

On écrit d'abord l'accélération :

\begin{equation}
\begin{aligned}
\mathbf{a}(t, \mathbf{x}) &=\lim _{\delta t \rightarrow 0} \frac{\mathbf{u}\left(t+\delta t, \mathbf{x}+\mathbf{u}(t, \mathbf{x}) \delta t+\mathcal{O}\left(\delta t^{2}\right)\right)-\mathbf{u}(t, \mathbf{x})}{\delta t} \\
&=\left(\frac{\partial \mathbf{u}}{\partial t}+\mathbf{u} \cdot \nabla \mathbf{u}\right)(t, \mathbf{x})
\end{aligned}
\end{equation}
En effet au temps $t + \delta t$ le fluide se sera déplacé dans la direction de sa vitesse et d'une longer égale à $\mathbf{u}\times \delta t$.


Maintenant, intéressons nous aux forces que subit le fluide. 
On en retiens généralement 2 : la force de gravité et les forces internes aux fluide qu'il exècre sur lui même.

La force de gravité : $\mathbf{F} = m \mathbf{g} = \rho V \mathbf{g}$, si on intègre sur notre volume $\mathcal{V}$, on obtient : 
\begin{equation}
\int_{\mathcal{V}} \rho(t, \mathbf{x}) \mathbf{g} \mathrm{d} x
\end{equation}

On écrit maintenant la force interne du fluide sur lui même : 
\begin{equation}
\int_{\partial \mathcal{V}} \sigma_{\mathrm{tot}}(t, \mathbf{x}) \mathbf{n}(\mathbf{x}) \mathrm{d} s
\end{equation}
avec $\sigma_{\mathrm{tot}}(t, \mathbf{x})$ le tenseur des contraintes du fluide.\\ C'est un tenseur d'ordre 2, donc pour nous une matrice $3\times3$ dans $\mathbf{R}^3$, $2\times2$ dans $\mathbf{R}^2$ qui est toujours symétrique.
Sa divergence est définie comme le vecteur dont les 3 composantes sont les divergences des vecteurs lignes du tenseur.

Finalement on a donc : 

\begin{equation}
\int_{\mathcal{V}} \rho(t, \mathbf{x})\left(\frac{\partial \mathbf{u}}{\partial t}+\mathbf{u} \cdot \nabla \mathbf{u}\right)(t, \mathbf{x}) \mathrm{d} x=\int_{\mathcal{V}} \rho(t, \mathbf{x}) \mathbf{g} \mathrm{d} x+\int_{\partial \mathcal{V}} \sigma_{\mathrm{tot}}(t, \mathbf{x}) \mathbf{n}(\mathbf{x}) \mathrm{d} s
\end{equation}

Comme précédemment on applique le théorème de Stokes au deuxième terme du second membre de l'égalité.
\[\int_{\partial \mathcal{V}} \sigma_{\mathrm{tot}}(t, \mathbf{x}) \mathbf{n}(\mathbf{x}) \mathrm{d} s  = \int_{\mathcal{V}} div(\sigma_{\mathrm{tot}}) \mathrm{d} x \]

D'où l'équation intégrale de la seconde loi de Newton appliquée au fluide : 
\begin{equation}
\int_{\mathcal{V}} \rho(t, \mathbf{x})\left(\frac{\partial \mathbf{u}}{\partial t}+\mathbf{u} \cdot \nabla \mathbf{u}\right)(t, \mathbf{x}) - div(\sigma_{\mathrm{tot}}) - \rho \mathbf{g} \, \, \mathrm{d} x = 0
\end{equation}
Par le même raisonnement que précédemment, on peut écrire une version locale :
 \begin{equation}\label{PreNav}
\rho(t, \mathbf{x})\left(\frac{\partial \mathbf{u}}{\partial t}+\mathbf{u} \cdot \nabla \mathbf{u}\right)(t, \mathbf{x}) - div(\sigma_{\mathrm{tot}}) - \rho \mathbf{g}  = 0
\end{equation}

\subsubsection{Tenseur de contrainte - Fluide Newtonien}

Un tenseur $\sigma$ peut être décomposée en 2 partie : 
\begin{equation}
\sigma = \underbrace{\left(\frac{1}{3} tr(\sigma)\right)}_{\text{Partie sphérique}}\times I_3+ \underbrace{\left(\sigma -  \frac{1}{3} tr(\sigma) \right)}_{\text{Partie déviatrice}}I_3
\end{equation}

Or on a un lien, pour les fluides isotropes, entre la pression hydrostatique et la parite sphérique du tenseur de contraintes : 

\begin{equation}
p = -\frac{tr(\sigma)}{3}
\end{equation}

On fait maintenant une autre hypothèse : le fluide est Newtonien. Cela signifie que la relation entre la vitesse de déformation et les contraintes du tenseur déviateur est linéaire.

cela ce traduit mathématiquement par : 
\begin{equation}\label{Tensor}
\underbrace{\left(\sigma -  \frac{1}{3} tr(\sigma) \right)}_{\text{Partie déviatrice}}I_3 = 2 \eta D(u) - \frac{2 \eta}{3}div(\mathbf{u})I_3
\end{equation}
$D(u)$e est le symétrisé du gradient de u : 
\begin{equation}
D(u) = \frac{\nabla u + \nabla u^T}{2}
\end{equation}

\subsubsection{Fluide incompressible}
On fait une dernière hypothèse, qui est que le fluide est incompressible, c'est à dire que la densité est constante dans le temps et l'espace, si on reprend l'équation \ref{MassConsLoc}, on obtient : 

\begin{equation}\label{Divnul}
\begin{aligned}
&\frac{\partial\rho}{\partial t} = -div(\rho \mathbf{u})  \\
\implies  & 0 = - \rho div(u) \\
\implies  & div(u) = 0 \\
\end{aligned}
\end{equation}
On dit que le champs de vitesse est à divergence nulle.

On remarque immédiatement que le dernier terme de l'équation \ref{Tensor} est nulle.
On a donc : 
\begin{equation}
\sigma_{\mathrm{tot}} = 2\eta D(u)
\end{equation}

Si on reprend l'équation \ref{PreNav} avec les nouvelles simplifications, on obtient : 
\begin{equation}
\rho\left(\frac{\partial \mathbf{u}}{\partial t}+\mathbf{u} \cdot \nabla \mathbf{u}\right)-\operatorname{div}(2 \eta D(\mathbf{u}))+\nabla p=\rho \mathbf{g}
\end{equation}

Enfin, on utilise l'identité suivante : 
\[\operatorname{div}(2D(\mathbf{u})) = \Delta \mathbf{u} + \nabla(\operatorname{div}(\mathbf{u}))\]
On a alors : 
\begin{equation}\label{NavPrincipal}
\rho\left(\frac{\partial \mathbf{u}}{\partial t}+\mathbf{u} \cdot \nabla \mathbf{u}\right)-\eta \Delta \mathbf{u} +\nabla p=\rho \mathbf{g}
\end{equation}

C'est l'équation principale de Navier-Stokes pour un fluide Newtonien, isotrope et incompressible.
On ajoute l'équation \ref{Divnul} pour avoir l’incompressibilité.
\subsubsection{Condition initiale et aux bords}

Étant donné que le problème de Navier-Stokes est un problème d'évolution, il faut se donner une condition initiale, c'est à dire connaître $\mathbf{u}$ au temps initiale $t_0$, appelons cette fonction $\mathbf{u}_0$.

Nous nous sommes donné un ouvert $\Omega$ dans lequel le problème est posé, il est donc nécessaire d'indiquer le comportement du fluide aux bord de $\Omega$, on not $\mathbf{u}_{\Gamma}$ cette fonction définie sur $\partial \Omega$.

On peut alors énoncer le Système de Navier-Stokes : 

\begin{equation}\label{SystNavStokes}
\left\{\begin{aligned}
\rho\left(\frac{\partial \mathbf{u}}{\partial t}+\mathbf{u} \cdot \nabla \mathbf{u}\right)-\eta \Delta \mathbf{u}+\nabla p &=\rho \mathbf{g} \text { dans }] 0, T[\times \Omega\\
-\operatorname{div}(\mathbf{u})&=0 \quad \text { dans }] 0, T[\times \Omega\\
\mathbf{u}(t=t_0) &=\mathbf{u}_{0} \text { dans } \Omega \\
\mathbf{u} &=\mathbf{u}_{\Gamma} \text { sur }] 0, T[\times \partial \Omega
\end{aligned}\right.
\end{equation}

\subsection{Problème de Stokes}
\subsubsection{Formulation forte}
Si on néglige le terme non linéaire $\mathbf{u} \cdot \nabla \mathbf{u}$ de l'équation de Navier-Stokes, on obtient l'équation de Stokes. Lorsque l'écoulement du fluide est lent, autrement dit quand le nombre de Reynolds est faible (inférieur à 1), alors cette approximation est valide. 
Le système de Stokes stationnaire générale s'écrit donc : 
\begin{equation}
\left\{\begin{aligned}
\alpha \mathbf{u}-\eta \Delta \mathbf{u}+\nabla p &=& \rho \mathbf{g} & \text { dans } \Omega \\
-\operatorname{div} \mathbf{u}  &=&0 & \text { dans } \Omega \\
\mathbf{u} &=& \mathbf{u}_{\Gamma} & \text { sur } \partial \Omega
\end{aligned}\right.
\end{equation}
On verra que l'on prend cette formulation en particulier car elle est utile pour résoudre le problème de Navier-Stokes par la méthode des caractéristiques.

\subsubsection{Formulation faible}
La résolution d'une équation différentielle par une méthode de type éléments fini commence toujours par la formulation faible de l'équation à résoudre.
On se donne donc une fonction test $\mathbf{v}$, on précisera dans quel espace fonctionnelle on prend cette fonction. Disons pour l'instant que la fonction est suffisamment régulière pour effectuer les opérations de dérivation et d'intégrations qui suivent. \\
On a donc : 
\begin{equation}
\int_{\Omega} \alpha \mathbf{u} \cdot \mathbf{v} \mathrm{d} x-\int_{\Omega} \eta \Delta \mathbf{u}\cdot \mathbf{v} \mathrm{d} x+\int_{\Omega} \nabla p \cdot \mathbf{v} \mathrm{d} x=\int_{\Omega} \rho  \mathbf{g}. \mathbf{v} \mathrm{d} x
\end{equation}
Et, en prenant une deuxième fonction test, $q$ : 
\begin{equation}
\int_{\Omega} q \operatorname{div} (\mathbf{u}) \mathrm{d} x = 0
\end{equation}
Premièrement, en appliquant la formule de Stokes \ref{StokesFormula} au produit $q \mathbf{v}$, on obtient : 
\begin{equation}
\int_{\Omega} \nabla q \cdot \mathbf{v}=-\int_{\Omega} q \operatorname{div}(\mathbf{v})+\int_{\partial \Omega} q \mathbf{v} \cdot \mathbf{n}
\end{equation}
Donc si on prend q = p, on a simplement : 
\begin{equation}
\int_{\Omega} \nabla p \cdot \mathbf{v}=-\int_{\Omega} p \operatorname{div}(\mathbf{v})+\int_{\partial \Omega} p \mathbf{v} \cdot \mathbf{n}
\end{equation}

Deuxièmement, en prenant $q=v_i$ et $\mathbf{v}$ par $\sigma_i$, on a, toujours par la formule de Stokes : 

\begin{equation}
\begin{aligned}
-\eta \int_{\Omega} \Delta \mathbf{u} \cdot \mathbf{v}dx &=\eta \int_{\Omega}\left(\nabla u_{1} \nabla v_{1}+\nabla u_{2} \nabla v_{2}\right)dx-\eta \int_{\partial \Omega}\left(v_{1}+v_{2}\right)\left(\nabla u_{1}+\nabla u_{2}\right) \cdot \mathbf{n} ds \\
&=\eta \int_{\Omega} \nabla \mathbf{u} \cdot \nabla \mathbf{v}dx-\eta \int_{\partial \Omega}\left(v_{1}+v_{2}\right)\left(\nabla u_{1}+\nabla u_{2}\right) \cdot \mathbf{n}ds
\end{aligned}
\end{equation}
 
 \subsubsection{Espaces fonctionnels}
On se demande quels sont les meilleurs espaces de fonctions dans lesquels prendre chercher la solution $(\mathbf{u}, p)$ et les fonctions tests $\mathbf{v}$ et $q$. Le but étant que les intégrales précédentes aient du sens, et que les espaces et les meilleurs propriétés possibles en terme de topologie etc.\\

Pour p, on prend l'espace des fonctions de carré intégrable et de moyenne nulle : 
\[
L_{0}^{2}(\Omega)=\left\{q \in L^{2}(\Omega) ; \int_{\Omega} q(x) \mathrm{d} x=0\right\}
\]
On ajoute la condition de moyenne nulle pour avoir unicité sur la pression, sinon elle ne serait unique qu'à une constante près.

Pour $\mathbf{u}$ et $\mathbf{v}$, on doit pouvoir au moins écrire la dérivée faible des fonctions, on part donc de l'espace de Sobolev $H^1$ : 

\begin{equation}
H^{1}(\Omega)=\left\{\varphi \in L^{2}(\Omega) ; \nabla \varphi \in L^{2}(\Omega)^{3}\right\}
\end{equation}

Pour u, on réstreint cet espace aux fonctions qui vérifie la condition aux bord :
\[H^{1}_{\Gamma}(\Omega)
\left\{\mathbf{v} \in H^{1}(\Omega)^{3} \mid \mathbf{v}=\mathbf{u}_{\Gamma} \operatorname{sur}\partial \Omega\right \}
\]
 
Et pour $\mathbf{v}$ on réstreint aux fonctions nulles sur le bord : 

\[H^{1}_{0}(\Omega)
\left\{\mathbf{v} \in H^{1}(\Omega)^{3} \mid \mathbf{v}=0 \operatorname{sur}\partial \Omega\right \}
\]

On a donc notre formulation faible finale : \\
Trouver $(\mathbf{u}, p) \in H^{1}_{\Gamma}(\Omega) \times L_{0}^{2}(\Omega)$ telle que pour tout 
$(\mathbf{v}, q) \in H^{1}_{0}(\Omega) \times L_{0}^{2}(\Omega)$, on ait : 

\begin{equation}
\begin{aligned}
\int_{\Omega}\alpha \mathbf{u} \cdot \mathbf{v}-\nabla \mathbf{u} \cdot \nabla \mathbf{v} \mathrm{d} x-& \int_{\Omega} p \operatorname{div}( \mathbf{v}) \mathrm{d} x=\int_{\Omega} \mathbf{f} \cdot \mathbf{v} \mathrm{d} x\\
&-\int_{\Omega} q \operatorname{div}( \mathbf{u}) \mathrm{d} x=0
\end{aligned}
\end{equation}

\subsubsection{Réécriture sous forme de système linéaire}
On introduit les opérateurs linéaires et bi-linéaires suivants : 
\begin{equation}
\begin{aligned}
a(\mathbf{u}, \mathbf{v}) &=\int_{\Omega}( \mathbf{u} \cdot \mathbf{v}-\eta \nabla \mathbf{u} \cdot \nabla \mathbf{v} ) \mathrm{d} x \\
b(\mathbf{v}, q) &=-\int_{\Omega} p \operatorname{div} (\mathbf{v}) \mathrm{d} x \\
l(\mathbf{v}) &=\int_{\Omega} \mathbf{f} \cdot \mathbf{v} \mathrm{d} x
\end{aligned}
\end{equation}

Le système de Stokes se réécrit alors : \\
Trouver $(\mathbf{u}, p) \in H^{1}_{\Gamma}(\Omega) \times L_{0}^{2}(\Omega)$ telle que pour tout 
$(\mathbf{v}, q) \in H^{1}_{0}(\Omega) \times L_{0}^{2}(\Omega)$, on ait : 

\begin{equation}
\left\{\begin{aligned}
a(\mathbf{u}, \mathbf{v})+b(\mathbf{v}, p) &=l(\mathbf{v}) \\
b(\mathbf{u}, q) &=0
\end{aligned}\right.
\end{equation}

