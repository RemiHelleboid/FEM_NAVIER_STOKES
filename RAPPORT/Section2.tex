\section{Méthodes des caractéristiques}
\subsection{Discrétisation temporelle}

On souhaite obtenir une solution numérique de l'équation \ref{SystNavStokes} sur l'intervalle de temps $[0, T]$, on prend donc $t_0 = 0$.
On a déjà montré la formule pour la dérivée totale d'une fonction à deux variable qui dépendent du temps : 
\begin{equation}
\frac{D \varphi}{D t}=\frac{\partial \varphi}{\partial t}+(\mathbf{u} \cdot \nabla) \varphi
\end{equation}
Ou $\mathbf{u}$ est le champs de vitesse associé à la variable $\mathbf{x}(t)$.
On définie la caractéristique de $\mathbf{x}(t)$ de la manière suivante : \\
On note $X^{(n)}(\mathbf{x}(t)) \in \mathbb{R}^{3}$  la position dont provient la particule qui est en 
$\mathbf{x}$ au temps $t_{n+1}$.
Un schéma d'Euler d'ordre 1 nous donne donc : 
\begin{equation}
\frac{D \varphi}{D t}\left(t_{n+1}, \mathbf{x}\right)=\frac{\varphi\left(t_{n+1}, \mathbf{x}\right)-\varphi\left(t_{n}, X^{(n)}(\mathbf{x})\right)}{\Delta t}+\mathcal{O}(\Delta t)
\end{equation}
Soit, appliquée au champs de vitesse $\mathbf{u}$ : 
\begin{equation}
\frac{D \mathbf{u}}{D t}\left(t_{n+1}, \mathbf{x}\right)=\frac{\mathbf{u}\left(\mathbf{x}, t_{n+1}\right)-\mathbf{u}\left(X^{(n)}(\mathbf{x}), t_{n}\right)}{\Delta t}+\mathcal{O}(\Delta t)
\end{equation}

Donc on peut réécrire l'équation principale de Navier-Stokes : \ref{NavPrincipal} 

\begin{equation}
\rho\left(\frac{\mathbf{u}\left(\mathbf{x}, t_{n+1}\right)-\mathbf{u}\left(X^{(n)}(\mathbf{x}), t_{n}\right)}{\Delta t}\right)-\eta \Delta \mathbf{u} +\nabla p=\rho \mathbf{g}
\end{equation}

On note $\mathbf{u}^{n}$ l'approximation de $\mathbf{u}(n\Delta t, x) $, on a alors le schéma implicite suivant : \\

\begin{equation}
\frac{\rho}{\Delta t} \mathbf{u}^{(n+1)}-\eta \Delta \mathbf{u}^{(n+1)}+\nabla p^{(n+1)}=\rho \mathbf{g}+\frac{\rho}{\Delta t} \mathbf{u}^{(n)} \circ X^{(n)}
\end{equation}

Toujours avec : 
\[div(\mathbf{u}^{(n+1)}) = 0\] 
\[\mathbf{u}^{(n+1)} = \mathbf{u}_{\Gamma} \text { sur } \partial \Omega\]

L'algorithme de résolution est donc le suivant : 

\begin{enumerate}
\item On se donne une solution initiale $\mathbf{u}^{(0)}$ qui respecte la condition de divergence nulle et la condition aux bords.
\item On résout, en utilisant une méhode d'éléments finis l'équation : \begin{equation*}
\frac{\rho}{\Delta t} \mathbf{u}^{(n+1)}-\eta \Delta \mathbf{u}^{(n+1)}+\nabla p^{(n+1)}=\rho \mathbf{g}+\frac{\rho}{\Delta t} \mathbf{u}^{(n)} \circ X^{(n)}
\end{equation*}
\[div(\mathbf{u}^{(n+1)}) = 0\] 
\[\mathbf{u}^{(n+1)} = \mathbf{u}_{\Gamma} \text { sur } \partial \Omega\]
\item On répète jusqu'à $n = \frac{T}{\Delta t}$.
\end{enumerate}

On se retrouve donc à devoir résoudre le problème de Stokes plusieurs fois. Or dans la section précédente on a déjà écris le système de Stokes sous sa forme faible elle même écrite avec applications linéaires 


\subsubsection{Réécriture de Stokes sous forme de système linéaire discret (Système matrice vecteur)}

Soient $X_{h} \subset H^{1}(\Omega)^{3}$ et $Q_{h} \subset L^{2}(\Omega)$ deux espaces vectoriels de dimension finie et soit $\mathbf{u}_{h, \Gamma}$ une approximation de $\mathbf{u}_{\Gamma},$ par exemple l'interpolée de Lagrange des données aux bords. Introduisons $V_{h},$ le sous espace de $X_{h}$ incluant les conditions aux limites :
$$
V_{h}\left(\mathbf{u}_{h, \Gamma}\right)=\left\{\mathbf{v}_{h} \in X_{h} ; \mathbf{v}_{h \mid \partial \Omega}=\mathbf{u}_{h, \Gamma}\right\}
$$$$
\begin{array}{l}
(F V S)_{h}: \text { trouver } \mathbf{u}_{h} \in V_{h}\left(\mathbf{u}_{h, \Gamma}\right), p_{h} \in Q_{h} \text { et } \lambda \in \mathbb{R} \text { tels que } \\ 
\\

\qquad\left\{\begin{aligned}
a\left(\mathbf{u}_{h}, \mathbf{v}_{h}\right)+b\left(\mathbf{v}_{h}, p_{h}\right) &=\left(\mathbf{f}, \mathbf{v}_{h}\right) \\
b\left(\mathbf{u}_{h}, q_{h}\right) =0 
\end{aligned}\right.
\end{array}
$$\\

pour tout $\mathbf{v}_{h} \in V_{h}(0), q_{h} \in Q_{h}$ et $\mu \in \mathbb{R} .$ 

\subsubsection{Résolution numérique}
\begin{equation}
\begin{aligned}
A &=\left(A_{i j}\right), \quad i, j\in [1, \cdots, dim(V_{h})]^2 \\
A_{i j}&=\int_{\Omega} \nabla \phi_{j} \cdot \nabla \phi_{i}\\
\mathbf{B} &=\left(B x_{i j}, B y_{i j}\right), \quad  i\in[1, \cdots, dim(Q_{h})]   \quad j\in[1, \cdots, dim(V_{h})] \\
 B x_{i j}&=-\int_{\Omega} \frac{\partial \phi_{j}}{ \partial x \varphi_{i}} \\
 B y_{i j}&=-\int_{\Omega} \frac{\partial \phi_{j} }{\partial y \varphi_{i}} \\
\end{aligned}
\end{equation}

On a alors le système linéaire : 
\begin{equation}
\left(\begin{array}{cc}
\mathbf{A} & \mathbf{B}^{T} \\
\mathbf{B} & 0
\end{array}\right)\left(\begin{array}{c}
\mathbf{U}_{h} \\
\left\{p_{h}\right\}
\end{array}\right)=\left(\begin{array}{c}
\mathbf{F}_{h} \\
0
\end{array}\right)
\end{equation}

\subsubsection{Méthode de pénalisation}

On utilise cette méthode pour avoir un meilleur système linéaire à résoudre, cela permet d'être sur que la méthode va converger vers une bonne solution et cela permet également que le système soit numériquement plus facile à résoudre.\\


On cherche à résoudre : 

\begin{equation}
\begin{aligned}
a\left(\mathbf{u}_{h}^{\epsilon}, \mathbf{v}_{h}\right)+b\left(\mathbf{v}_{h}, p_{h}^{\epsilon}\right) &=\left(\mathbf{f}, \mathbf{v}_{h}\right), \quad \forall \mathbf{v}_{h} \in \mathbf{V}_{h} \\
b\left(\mathbf{u}_{h}^{\epsilon}, q_{h}\right)-\epsilon\left(p_{h}^{\epsilon}, q_{h}\right) &=0, \quad \forall q_{h} \in Q_{h}
\end{aligned}
\end{equation}

Toujours avec : $(\mathbf{u}_{h}^{\epsilon}, p_{h}^{\epsilon}) \in V_h \times Q_h$\\


Donc en écriture matricielle : 

\begin{equation}
\left(\begin{array}{cc}
\mathbf{A} & B^{*} \\
B & -\epsilon I
\end{array}\right)\left(\begin{array}{c}
\mathbf{U}_{h}^{\epsilon} \\
\left\{p_{h}^{\epsilon}\right\}
\end{array}\right)=\left(\begin{array}{c}
\mathbf{F}_{h} \\
0
\end{array}\right)
\end{equation}

Il existe d'autre méthode pour résoudre le problème de Stokes mais on utilise toujours une méthode plus complexe que simplement résoudre (36).